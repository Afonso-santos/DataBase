\documentclass[a4paper,12pt]{scrreprt}
    %% Used for changing geometry of the page
    %% Cover page text cannot overlay cover sketching/style
    %% https://ctan.org/pkg/geometry?lang=en
\usepackage{geometry}
    %% Changes language of some packages protocols
    %% e.g., when captioning images: Figure 1. -> Figura 1.
    %% https://ctan.org/pkg/babel?lang=en
\usepackage[portuguese]{babel}
    %% Used for special fonts
    %% Cannot be compiled with pdflatex
    %% https://ctan.org/pkg/fontspec?lang=en
\usepackage{fontspec}
    %% Arial FONT
    \setmainfont{Arial}

    %% More colors and color options
    %% https://ctan.org/pkg/xcolor?lang=en
    %% https://ctan.org/pkg/colortbl?lang=en
\usepackage{xcolor,colortbl}
    %% More tabular options, like dashed/dotted lines
    %% https://ctan.org/pkg/arydshln?lang=en
\usepackage{arydshln}
    %% List of acronyms
    %% https://ctan.org/pkg/nomencl?lang=en
\usepackage[intoc]{nomencl}
    %% Must be called to init nomencl environment
    \makenomenclature
    %% More images options/settings
    %% https://ctan.org/pkg/graphicx?lang=en
\usepackage{graphics}
    %% Defining subdirectories to image path enviornment
    %% \graphicspath{{sub1}{sub2}...{subN}}
    \graphicspath{{images}}

    %% used to handle cross-referencing commands in LaTeX to produce hypertext links in the document
    %% https://ctan.org/pkg/hyperref?lang=en
\usepackage{hyperref}
    %% math environments
    %% https://ctan.org/pkg/amsmath?lang=en

    %% settings
    \hypersetup{
        colorlinks,
        citecolor=black,
        filecolor=black,
        linkcolor=black,
        urlcolor=black
    }

\usepackage{amsmath}
    %% Defining backgrouns, used to make the cover
    %% https://ctan.org/pkg/background?lang=en
\usepackage[some]{background}
    %% Used to make drawings or complex graphics
    %% http://pgf.sourceforge.net/pgf_CVS.pdf
\usepackage{tikz}
    %% Tikz library to point operations ((x1,y1) + (x2,y2))
    \usetikzlibrary{calc}

%% Defining sfdefault font and default font for document
\renewcommand{\familydefault}{\sfdefault}

%==========================================================================
% DOCUMENT
%==========================================================================

\begin{document}

\pagenumbering{gobble}

%% Costume made cover
%% From there you can use \makecover command to build the cover
\include{cover}

% builds the cover
\makecover

%% smaller footer and header size
\newgeometry{top=3cm,left=3cm,right=3cm,bottom=4cm}

%==========================================================================
% BEGIN OPCIONAL DEDICATÓRIA
%==========================================================================

% \clearpage
% \begin{center}
%     \thispagestyle{empty}
%     \vspace*{\fill}
%
%     $<<$/opcional Dedicatória$>>$
%
%     \vspace*{\fill}
% \end{center}
% \clearpage

%==========================================================================
% END OPCIONAL DEDICATÓRIA
%==========================================================================

%==========================================================================
% BEGIN ABSTRACT PAGE
%==========================================================================

%% Abstract name: \Large font size, flushed left and paragraph skip before abstract content
\renewenvironment{abstract}
 {\par\noindent\textbf{\Large\abstractname}\par\bigskip}
 {}

\begin{flushleft}
\begin{abstract}
    No âmbito da UC Base de Dados, lecionada pelo regente, Professor Orlando Belo, visamos a
    realização de um projeto que consiste na modelação, desenvolvimento e implementação de um
    Sistema de Base de Dados.
    \par Tendo em consideração o tema deste ano - uma agência de detetives -, decidimos explorar
    uma agência liderada por Agatha Christie, de nome CDC - Consultoria de Detetives Christie.
    Relativamente à modelação e desenvolvimento do nosso projeto, iremos dividi-lo em partes:
    começaremos de forma abstrata, transformando os factos
    \par \textbf{Área de Aplicação}: \textcolor{red}{
        <<Identificação da Área de trabalho. Por exemplo: Desenho e arquitectura de Sistemas de Bases de Dados.>>
    }
    \par \textbf{Palavras-Chave}: \textcolor{red}{
        <<Conjunto de palavras-chave que permitirão referenciar domínios de conhecimento, tecnologias, estratégias, etc., directa ou indirectamente referidos no relatório. Por exemplo: Bases de Dados Relacionais, Gestão de Índices, JAVA, Protocolos de Comunicação.>>
    }
\end{abstract}
\end{flushleft}

\pagebreak

%==========================================================================
% END ABSTRACT PAGE
%==========================================================================

%==========================================================================
% BEGIN INDEXES PAGES
%==========================================================================

%% Changes table of content name
%% Portuguese babel default : "Conteúdo"
%% Personally I prefer "índice"
\renewcommand{\contentsname}{Índice}
\renewcommand{\listfigurename}{Índice de Figuras}
\renewcommand{\listtablename}{Índice de Tabelas}

\tableofcontents

\pagebreak

\listoffigures

\pagebreak

\listoftables

\pagebreak

%==========================================================================
% END INDEXES PAGES
%==========================================================================

%==========================================================================
% BEGIN INTRODUCTION
%==========================================================================

%% Starting page numbering here
\pagenumbering{arabic}

\chapter{Definição do Sistema}
    \section{Contexto de Aplicação}
    Agatha Christie, uma figura proeminente no mundo dos detetives, criou a sua própria agência
    no final dos anos 90 após concluir que a sua carreira como detetive privada não ia ser
    suficiente para vingar-se do mundo sujo e curioso do crime.
    \par A sua agência começou como algo discreto - um escritório na periferia de Londres,
    constituído por Agatha - gerente e secretária, a cara da Consultoria de Detetives Christie
    (CDC) -, e mais três detetives, responsáveis por resolver os casos dos clientes que recorriam
    à agência nos seus momentos de aflição.
    \par Não obstante, nos últimos três anos, houve um crescimento exponencial de casos, visto que a
    sua agência tornou-se renomada devido a alta variedade de casos que é capaz de solucionar -
    desde os mais “mundanos”, como casos de infidelidade e perseguições, até aos mais
    “mórbidos”, como homicídios e desaparecimentos. E, visto que Agatha é fascinada pelo avanço
    tecnológico, a sua consultoria também é exemplo de vanguarda na solução de cibercrimes.
    \par Como tal, toda esta nova popularidade acrescida levou a que Agatha contratasse um novo
    estagiário, aumentando a sua equipa e procurando conseguir prepará-lo para a subida de
    casos que a agência enfrentava a todo o vapor.
    \section{Motivação e Objetivos do Trabalho}
    A CDC enfrenta um aumento significativo na demanda pelos seus serviços de investigação devido à
    sua reputação crescente e à diversificação dos casos que lida. Infelizmente, Agatha sentiu a sua
    valiosa agência a sofrer complicações a partir do momento em que decidiram aceitar um maior número
    de casos - O aumento na procura por serviços de investigação levou a uma sobrecarga nos sistemas de
    gerenciamento de casos existentes, e os registos físicos que ela mantinha desde o início da sua
    agência não lhe permitiam atribuir com rapidez suficiente os seus detetives aos casos, e muitas das
    informações cruciais, como pistas ou relatos de testemunhas, já haviam sido perdidos ou duplicados
    no passado, o que fazia Agatha temer que a sua agência acabasse por ficar com uma má reputação.
    \par Com a sua mente analítica e perspicaz, ela reconheceu que a chave para resolver este mistério
    organizacional estava  na modernização tecnológica, nomeadamente, na implementação de um Sistema de
    Base de Dados que possa lidar com a crescente quantidade de informações e casos de forma eficiente e
    escalável, bem como gerenciar e organizar as informações relacionadas aos casos, clientes, evidências
    e suspeitos. Este projeto visa atender a essa demanda e proporcionar à CDC as ferramentas necessárias
    para continuar a oferecer serviços de alta qualidade e eficácia, assim garantindo o sucesso da agência
    e aliviando as preocupações de Agatha sobre a popularidade acrescida.
    \par Por conseguinte, os objetivos principais que pretendemos alcançar com o desenvolvimento deste SBD são os seguintes:
    \begin{itemize}
        \item \textbf{Escalabilidade:} À medida que a CDC cresce e enfrenta um aumento contínuo na demanda pelos
            seus serviços, é essencial ter um sistema que possa escalar para atender às necessidades em constante
            evolução da agência. Um Sistema de Base de Dados escalável pode crescer junto com a CDC, garantindo que
            ela permaneça ágil e adaptável às mudanças no mercado, sem falhas ou confusões no sistema.
        \item \textbf{Acesso Rápido e Eficiente de Dados:} Com um Sistema de Base de Dados, os dados relacionados a
            casos, clientes, evidências e investigações podem ser acessados rapidamente num local centralizado de
            forma rápida e eficiente, o que permite uma colaboração mais eficaz entre os detetives e facilita a
            tomada de decisões informadas.
        \item \textcolor{red}{
                \textbf{Eficiência Operacional:} Com o aumento do volume de casos, os métodos manuais
                de organização de informações tornaram-se cada vez mais ineficientes. Um Sistema de Base de Dados pode
                automatizar diversas tarefas, como armazenamento, recuperação e atualização de dados, libertando tempo
                e recursos dos funcionários para se concentrarem em investigações mais complexas.
            }
        \item \textbf{Precisão e Consistência:} Os registos físicos estão sujeitos a erros humanos, como duplicação de
            dados ou informações desatualizadas. Um Sistema de Base de Dados garante precisão e consistência nas
            informações, ajudando assim a evitar erros e inconsistências que possam comprometer a qualidade do
            trabalho da CDC.
        \item \textbf{Segurança de dados:} Os registos físicos podem ser facilmente acessados por qualquer pessoa
            que os encontre. Isso inclui funcionários não autorizados ou intrusos, provocando falsificações, destruição
            acidental e intencional de provas. Com a implementação de Sistema de Bases de Dados, existe um maior
            controlo de acesso.
    \end{itemize}
    \section{Análise da Viabilidade do Processo}
        A viabilidade de um projeto de desenvolvimento de software depende da habilidade de compreender e satisfazer as
        demandas do mercado e dos utilizadores. Isto requer um planeamento cuidadoso e eficiente para garantir a
        entrega de um produto confiável e de alta qualidade. E, ao seguir uma abordagem metódica, o projeto pode
        maximizar as suas chances de sucesso ao atender às expectativas e necessidades do público-alvo de forma eficaz.
        \par Considerando agora a viabilidade do nosso projeto, acreditamos que seja bastante viável, pois irá garantir
        uma série de benefícios para a agência, nomeadamente:
        \begin{itemize}
            \item \textbf{Melhor Gestão de Funcionários:} Com um Sistema de Base de Dados, existe uma maior facilidade
                para identificar que funcionários estão ocupados ou disponíveis, possibilitando uma alocação mais
                rápida dos mesmos aos casos e uma melhor assistência destes conforme necessária.
            \item \textbf{Melhorar a Qualidade de Serviço e de Bem Estar no Trabalho:} Um Sistema de Bases de Dados
                promoverá um melhor bem estar aos seus funcionários, evitando buscas intensivas ao sistema de
                informação já recolhida, consequentemente melhorando a qualidade do serviço, significativamente.
            \item \textbf{Resolver a Sobrecarga:} Devido aos dois pontos referidos anteriormente, os funcionários
                serão capazes de resolver um caso com mais eficiência e rapidez, ficando disponíveis mais rapidamente.
                Como tal, a sua produtividade vai aumentar e vai ficar a par da nova enchente de casos.
            \item \textbf{Segurança Acrescida:} Com um Sistema de Base de Dados, existe um maior controlo de acesso
                relativamente a informações cruciais aos casos, o que garante a inexistência de adulteração ou
                destruição de provas. Com isto, tem-se a certeza que as informações presentes nos registos são as
                originais e não foram acedidas por intrusos.
        \end{itemize}
        \par Considerando esses fatores, fica claro que o projeto de implementação do Sistema de Base de Dados é
        altamente viável e trará benefícios substanciais para a CDC, principalmente a níveis financeiros, de
        organização de dados e serviços, e, a longo prazo, de crescimento contínuo no mercado de detetives particulares.
    \section{Recursos e Equipa de Trabalho}
        \subsection{Recursos Humanos:}
            \begin{itemize}
                \item Funcionários da Consultoria (Detetives, estagiários e gerente);
                \item Clientes (Vítimas e os seus familiares, etc);
                \item Equipa de desenvolvimento.
            \end{itemize}
        \subsection{Recursos Físicos:}
            \begin{itemize}
                \item Computadores;
                \item Conexão à \textit{Internet};
                \item Servidor.
            \end{itemize}
        \subsection{Recursos Digitais:}
            \begin{itemize}
                \item Sistemas Operativos: \textit{Windows} 11 e \textit{Linux} (\textit{Ubuntu} 22.04.3 \textit{LTS});
                \item \textit{Google Drive};
                \item \textit{GitHub};
                \item \textit{LaTeX};
                \item brModelo (v3.31);
                \item \textcolor{red}{(...) vamos preenchendo isto à medida que vamos usando ferramentas}.
            \end{itemize}
        \subsection{Equipa de Trabalho:}
            \begin{itemize}
                \item \textbf{Perssoal Interno:}
                \begin{itemize}
                    \item \textbf{Agatha Christie:} Funcionamento da agência, atendimento a clientes,
                        validação de serviços, atribuição de casos aos agentes, depoimento de informações
                        cruciais ao projeto.
                    \item \textcolor{red}{\textbf{Detetives efetivos:}}
                    \item \textcolor{red}{\textbf{Detetives estagiários:}}
                \end{itemize}
            \item \textbf{Pessoal Externo:} \textcolor{red}{(para continuar ao longo do tempo)}
                \begin{itemize}
                    \item \textbf{Afonso Santos:}
                    \item \textbf{Ana Pinto:}
                    \item \textbf{Carlos Ferreira:} Levantamento de Requisitos
                    \item \textbf{Flávia Araújo:}
                    \item \textbf{Miguel Carvalho:} Levantamento de Requisitos
                \end{itemize}
            \end{itemize}

    \section{Plano de Execução do Projeto}
        \textcolor{red}{
            <<TODO: adicionar diagrama de Gantt>>
        }
    \section{Estrutura do Relatório}
        \textcolor{red}{
            <<Após a leitura da introdução de um relatório é "simpático" apresentar uma breve descrição daquilo que se vai encontrar nos demais capítulos do relatório.>>
        } \\
        \textcolor{red}{[isto faz-se no fim]}
%==========================================================================
% END INTRODUCTION
%==========================================================================

%==========================================================================
% BEGIN LEVANTAMENTO E ANÁLISE DE REQUISITOS
%==========================================================================

\chapter{Levantamento e Análise de Requisitos}
    \section{Método de Levantamento e de Análise de Requisitos Adotado}
        \textcolor{red}{
            >> Identificação, justificação e caracterização das diversas formas de levantamento de requisitos adotadas, expondo os seus diversos intervenientes. Referir e incluir eventual documentação utilizada ou recolhida – atas de reuniões, documentos de processos de trabalho, diagramas de atividade, etc.
        }
    \section{Organização dos Requisitos Levantados}
        \textcolor{red}{
            >> Explicação do processo de levantamento realizado e exposição da organização de requisitos adotada. Enumeração e organização dos requisitos levantados, de acordo com a sua categorização (descrição, manipulação e controlo). Caracterização dos requisitos tendo em consideração os seguintes elementos: tipo, número, data, descrição do requisito, fonte de informação e analista.
        }
    \section{Análise e Validação Geral dos Requisitos}
        \textcolor{red}{
            >> Validação dos requisitos apresentados com os diversos intervenientes do processo. Relatar eventuais anomalias e ações corretivas desenvolvidas.
        }

%==========================================================================
% END LEVANTAMENTO E ANÁLISE DE REQUISITOS
%==========================================================================

%==========================================================================
% BEGIN MODELAÇÃO CONCEPTUAL
%==========================================================================

\chapter{Modelação Conceptual}
    \section{Apresentação da Abordagem de Modelação Realizada}
        \textcolor{red}{
            <<Apresentação e explicação do processo de modelação concetual adotado, com referência à notação e ferramenta adotadas.>>
        }
    \section{Identificação e Caracterização das Entidades}
        \textcolor{red}{
            <<Explicação do processo de identificação das entidades do sistema. Enumeração e fundamentação de cada uma das entidades identificadas, tendo em conta a caracterização sugerida em [Connoly e Begg, 2015]. Indicar os requisitos que originaram cada uma das entidades.>>
        }
    \section{Identificação e Caracterização dos Relacionamentos}
        \textcolor{red}{
            <<Explicação do processo de identificação dos relacionamentos entre as entidades definidas. Enumeração e fundamentação de cada um dos relacionamentos identificados, tendo em conta a caracterização sugerida em [Connolly e Begg, 2015]. Indicar os requisitos que originaram cada um dos relacionamentos enumerados.>>
        }
    \section{Identificação e Caracterização dos Atributos das Entidades e dos Relacionamentos}
        \textcolor{red}{
            <<Explicação do processo de identificação dos vários atributos das entidades e dos relacionamentos. Enumeração e fundamentação de cada atributo, organizando-os por entidades e relacionamentos identificados, tendo em conta a caracterização sugerida em [Connolly e Begg, 2015]. Indicar os requisitos que originaram cada um dos atributos enumerados.>>
        }
    \section{Apresentação e Explicação do Diagrama ER Produzido}
        \textcolor{red}{
            <<Apresentação do diagrama concetual produzido. Explicação do seu processo de construção, tendo em conta as entidades e os relacionamentos identificados anteriormente, bem com as suas caracterizações em termos de atributos.>>
        }

%==========================================================================
% END MODELAÇÃO CONCEPTUAL
%==========================================================================

%==========================================================================
% BEGIN MODELAÇÃO LÓGICA
%==========================================================================

\chapter{Modelação Lógica}
    \section{Construção e Validação do Modelo de Dados Lógico}
        \textcolor{red}{
            <<Apresentação e explicação do processo de modelação lógica adotado, com referência à ferramenta adotada.>>
        }
    \section{Apresentação e Explicação do Modelo Lógico Produzido}
        \textcolor{red}{
            <<Apresentação do processo de conversão realizado, expondo e justificando a origem de cada uma das tabelas que foram criadas. Apresentação do modelo lógico produzido.>>
        }
    \section{Normalização de Dados}
        \textcolor{red}{
            <<Indicação se o modelo está ou não normalizado. Explicar.>>
        }
    \section{Validação do Modelo com Interrogações do Utilizador}
        \textcolor{red}{
            <<Apresentação de 4-6 expressões em Álgebra Relacional que representem queries previamente enunciadas no conjunto de requisitos de manipulação estabelecidos anteriormente.>>
        }

%==========================================================================
% END MODELAÇÃO LÓGICA
%==========================================================================

%==========================================================================
% BEGIN CONCLUSÕES DE TRABALHO FUTURO
%==========================================================================

\chapter{Conclusões e Trabalho Futuro}
    \textcolor{red}{
        <<Elaborar uma apreciação crítica sobre o trabalho realizado, apontando os seus pontos fortes e fracos. Adicionalmente, caso se aplique, enunciar eventuais tarefas a realizar futuramente ou novas opções para estender o trabalho realizado.>> \\
        <<Resumo breve do trabalho realizado e das ações desenvolvidas. Enumeração e análise de aspetos positivos e negativos identificados durante o processo de desenvolvimento do sistema de bases de dados. Exposição das próximas linhas de desenvolvimneto do projeto.>>
    }

%==========================================================================
% END CONCLUSÕES DE TRABALHO FUTURO
%==========================================================================

%==========================================================================
% BEGIN BIBLIOGRAFIA
%==========================================================================

%% Changes biblibography name
%% Portuguese babel default : "Bibliografia"
%% Personally I prefer "Referências"
% \renewcommand\bibname{Referências}

%% https://www.overleaf.com/learn/latex/bibliography_management_with_bibtex
\begin{thebibliography}{9}
\textcolor{red}{
    <<Enumeração dos diversos recursos bibliográficos utilizados. Utilizar o sistema de referenciação APA (https://guias.sdum.uminho.pt/apa).>>
}
\end{thebibliography}

%==========================================================================
% END BIBLIOGRAFIA
%==========================================================================

%==========================================================================
% BEGIN LISTA DE SIGLAS E ACRÓNIMOS
%==========================================================================

%% Portuguese babel does not translate this environment
\renewcommand{\nomname}{Lista de Siglas e Acrónimos}

%% Text that can be shown before acronyms list
% \renewcommand{\nompreamble}{
%     \textcolor{red}{
%         <<Apresentar uma lista com todas as siglas e acrónimos utilizados durante a realização do trabalho. O formato base para esta lista deverá ser da forma como abaixo se apresenta.>>
%     }
% }

%% acronyms
\nomenclature[01]{\textbf{CDC}}{Consultoria de Detetives Christie}
\nomenclature[02]{\textbf{BD}}{Base de Dados}
\nomenclature[03]{\textbf{SBD}}{Sistema de Base de Dados}
\nomenclature[04]{\textbf{ER}}{Entidade-Relacionamento}

%% Show acronyms
\printnomenclature

%==========================================================================
% END LISTA DE SIGLAS E ACRÓNIMOS
%==========================================================================

%==========================================================================
% BEGIN ANEXOS
%==========================================================================

%% Why \addchap, instead of \chapter?
%% \addchap has no numbering but appears in table of contents.
\addchap{Anexos}

    \textcolor{red}{
        <<Os anexos deverão ser utilizados para a inclusão de informação adicional necessária para uma melhor compreensão do relatório o para complementar tópicos, secções ou assuntos abordados. Os anexos criados deverão ser numerados e possuir uma designação. Estes dados permitirão complementar o Índice geral do relatório relativamente à enumeração e apresentação dos diversos anexos.>>
    }

    %% section version of \addchap
    \addsec{Anexo 1}


%==========================================================================
% END ANEXOS
%==========================================================================
\end{document}
